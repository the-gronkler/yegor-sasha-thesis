\section*{\centering \pjabstract}

\ifthenelse
    {\equal{\pjlanguage}{english}}
    {Shortly describe what kind of problem has been inspected in the thesis, and how it has been solved. The abstract should be between 400 and 1500 characters, including spaces. \textbf{Thesis written in English must also include abstract and keyword translations to Polish}. Abstract should usually be written \textbf{towards the end} of your thesis work, since that is the time when you best know what (and how) exactly has been accomplished. \textbf{Pay extra attention and spend some extra time when developing an abstract.} This is due to the fact that most people will be glancing over your abstract in order to determine whether it is worth it for them to delve deeper into your work. This is the place where you need to attractively explain what can be found in this thesis. Do not introduce additional paragraphs in the abstract. The rest of this document describes general rules for writing theses documentations in \pj.}
    {Krótko opisz, jaki problem został poruszony w pracy oraz w jaki sposób został rozwiązany. Streszczenie powinno zawierać od 400 do 1500 znaków, włączając w to spacje. Zwykle streszczenie najlepiej jest pisać \textbf{pod koniec} pracy, bo wtedy najlepiej wiemy, co ostatecznie udało się osiągnąć i jaka droga do tego prowadziła. \textbf{Bądź bardzo uważny i spędź trochę więcej czasu tworząc streszczenie.} To dlatego, że większość osób będzie powierzchownie zerkała na nie, gdy będą decydować, czy zagłębienie sie w szczegóły tej pracy jest warte ich czasu. W streszczeniu nie oddzielaj paragrafów. Reszta tego dokumentu opisuje ogólne zasady tworzenia prac dyplomowych w \pj. Instrukcje są pisane po angielsku, ale zmiana języka dokumentu zmieni wiele automatycznie generowanych elementów, takich jak: logo, nazwy sekcji typu Spis Treści, Streszczenie czy Słowa kluczowe.}

\section*{\centering \pjkeywords}

\ifthenelse
    {\equal{\pjlanguage}{english}}
    {Keywords \( \cdot \) can \( \cdot \) be \( \cdot \) both \( \cdot \) single- or multiple-word phrases \( \cdot \) At \( \cdot \) least \( \cdot \) 3 \( \cdot \) keywords \( \cdot \) are \( \cdot \) necessary \( \cdot \) Threat \( \cdot \) them \( \cdot \) as \( \cdot \) tags \( \cdot \) Your \( \cdot \) thesis \( \cdot \) must \( \cdot \) be \( \cdot \) searchable \( \cdot \) using \( \cdot \) them \( \cdot \) Separate them using the \lstinline{\\( \\cdot \\)} syntax.}
    {Słowa \( \cdot \) kluczowe \( \cdot \) mogą \( \cdot \) być \( \cdot \) zarówno \( \cdot \) pojedynczymi \( \cdot \) słowami \( \cdot \) jak i frazami \( \cdot \) Wymagane \( \cdot \) są \( \cdot \) co \( \cdot \) najmniej \( \cdot \) 3 \( \cdot \) słowa \( \cdot \) kluczowe \( \cdot \) Traktuj \( \cdot \) je \( \cdot \) jako \( \cdot \) tagi \( \cdot \) Praca \( \cdot \) musi \( \cdot \) być \( \cdot \) wyszukiwalna \( \cdot \) przy \( \cdot \) ich \( \cdot \) pomocy \( \cdot \) Oddzielaj je składnią \lstinline{\\( \\cdot \\)}.}

    

